%% start of file `template.tex'.
%% Copyright 2006-2013 Xavier Danaux (xdanaux@gmail.com).
%
% This work may be distributed and/or modified under the
% conditions of the LaTeX Project Public License version 1.3c,
% available at http://www.latex-project.org/lppl/.


\documentclass[12pt,a4paper,sans]{moderncv}        % possible options include font size ('10pt', '11pt' and '12pt'), paper size ('a4paper', 'letterpaper', 'a5paper', 'legalpaper', 'executivepaper' and 'landscape') and font family ('sans' and 'roman')

% moderncv themes
\moderncvstyle{classic}                             % style options are 'casual' (default), 'classic', 'oldstyle' and 'banking'
\moderncvcolor{blue}                               % color options 'blue' (default), 'orange', 'green', 'red', 'purple', 'grey' and 'black'
%\renewcommand{\familydefault}{\sfdefault}         % to set the default font; use '\sfdefault' for the default sans serif font, '\rmdefault' for the default roman one, or any tex font name
%\nopagenumbers{}                                  % uncomment to suppress automatic page numbering for CVs longer than one page

% character encoding
\usepackage[utf8]{inputenc}                       % if you are not using xelatex ou lualatex, replace by the encoding you are using
%\usepackage{CJKutf8}                              % if you need to use CJK to typeset your resume in Chinese, Japanese or Korean

% adjust the page margins
\usepackage[scale=0.75]{geometry}
%\setlength{\hintscolumnwidth}{3cm}                % if you want to change the width of the column with the dates
%\setlength{\makecvtitlenamewidth}{10cm}           % for the 'classic' style, if you want to force the width allocated to your name and avoid line breaks. be careful though, the length is normally calculated to avoid any overlap with your personal info; use this at your own typographical risks...

\usepackage{ragged2e}
% personal data
\name{Bruno}{Ximenez}
\title{Resumé title}                               % optional, remove / comment the line if not wanted
\address{28 Rue Godefroy-Cavaignac 75011}{}{Paris, France}% optional, remove / comment the line if not wanted; the "postcode city" and and "country" arguments can be omitted or provided empty
\phone[mobile]{+33 75 149 8033}                   % optional, remove / comment the line if not wanted
%\phone[fixed]{+2~(345)~678~901}                    % optional, remove / comment the line if not wanted
%\phone[fax]{+3~(456)~789~012}                      % optional, remove / comment the line if not wanted
\email{bruno.ximenez@obspm.fr}                               % optional, remove / comment the line if not wanted
%\homepage{www.johndoe.com}                         % optional, remove / comment the line if not wanted
%\extrainfo{additional information}                 % optional, remove / comment the line if not wanted
%\photo[64pt][0.4pt]{picture}                       % optional, remove / comment the line if not wanted; '64pt' is the height the picture must be resized to, 0.4pt is the thickness of the frame around it (put it to 0pt for no frame) and 'picture' is the name of the picture file
%\quote{Some quote}                                 % optional, remove / comment the line if not wanted

% to show numerical labels in the bibliography (default is to show no labels); only useful if you make citations in your resume
%\makeatletter
%\renewcommand*{\bibliographyitemlabel}{\@biblabel{\arabic{enumiv}}}
%\makeatother
%\renewcommand*{\bibliographyitemlabel}{[\arabic{enumiv}]}% CONSIDER REPLACING THE ABOVE BY THIS

% bibliography with mutiple entries
%\usepackage{multibib}
%\newcites{book,misc}{{Books},{Others}}
%----------------------------------------------------------------------------------
%            content
%----------------------------------------------------------------------------------
\begin{document}
%-----       letter       ---------------------------------------------------------
% recipient data
\recipient{Saint Gobain}{39 Quai Lucien Lefranc, 93300 \\Aubervilliers\\ France}
\date{Juillet 26, 2020}
\opening{Madame, Monsieur,}
\closing{Dans l’attente d’une réponse de votre part, je vous prie d’agréer, Madame, Monsieur, l’expression de mes sentiments distingués.}
\enclosure[Attached]{curriculum vit\ae{}}          % use an optional argument to use a string other than "Enclosure", or redefine \enclname
\makelettertitle
\justifying

Actuellement chercheur en contrat post-doctoral au SYRTE situé à l'Observatoire de Paris depuis le mois de juin 2018, je suis à la recherche d’une collaboration dans le domaine des technologies.

Mon travail de recherche post-doctoral est majoritairement fondé sur la métrologie du temps et des fréquences optiques, grâce à laquelle la seconde sera redéfinie dans le futur proche. Je  contribue également au développement de méthodes de mesure dans le but d’améliorer la précision et la stabilité du standard de fréquence optique strontium.

Ayant en doctorat en physique orienté sur les mesures de haute précision, j’ai eu l’occasion, au cours de ma carrière professionnelle et dans le cadre de mes recherches, de résoudre de nombreux problèmes s’étendant au-delà de la physique fondamentale, au moyen de la conception mécanique et électronique, et de la conception d’algorithmes de logiciels pour l'analyse des données et pour la manipulation au niveau du hardware. Les résultats que j’ai obtenu en matière de physique fondamentale ont été publiés dans Nature, le journal scientifique ayant le paramètre d'impact le plus élevé dans le monde.  

Je suis passionné par l'apprentissage que la science et la technologie m'ont offert, et je me sens par conséquent prêt à relever un nouveau défi, où je pourrai à la fois contribuer rapidement aux projets proposés par Saint Gobain Recherche, tout en élargissant mes compétences et mes connaissances aussi bien sur le plan technique qu’humain. 

Pour les raisons exposées ci-avant, les opportunités qu’offrent le poste d’Ingénieur-Chercheur au sein de Saint Gobain suscitent donc particulièrement mon interêt, d’autant plus que celles-ci me donneront la chance de contribuer activement et directement à la société.

% I recently finished my PhD program on antimatter research developed at the Antiproton Decelecetor (AD) hall, located at the Organisation européenne
% pour la recherche nucléaire (CERN). The goal consisted in measuring the line shape of the 1S--2S transition in antihydrogen and determining the associated resonance frequency and,furthermore the consequences of the result in the frame of our current understanding of nature. The research was reported in five articles published along the PhD program. 

% I have learned and devolped a number of advanced techniques on modern laser spectroscopy to allow the measurment of the atomic resonance, which consisted briefly of optical metrology, in particular, laser frequency measurements with a referenced comb, Pound--Drever--Hall lock, second harmonic generation, enhacement cavity on cryogenic environment. Other important techniques devolped were on basic electronics design (analog and digital), design of mechanical parts for vacuum systems, vacuum and cryogenic technologies.         

% I am now in position of contributing with your experiment by bringing my current knowledge gathered through my education process and at the same time, expand my professional skills by learning deeply about optical metrology with strontium atoms stored in an optical lattice. 

\makeletterclosing

\end{document}


%% end of file `template.tex'.
